\documentclass[12pt,a4paper]{article}
\usepackage[latin1]{inputenc}
\usepackage{amsmath}
\usepackage{amsfonts}
\usepackage{amssymb}
\usepackage{makeidx}
\usepackage{graphicx}
\usepackage[left=2cm,right=2cm,top=2cm,bottom=2cm]{geometry}

\begin{document}
\title{Universidad Politecnica de la Zona Metropolitana de Guadalajara}
\author{Tarea 2 \\ Josue Natanael Orozco Nevares 18311797 \\ Ing. Mecatronica 4B}
\maketitle

\begin{figure}[h!]
\centering
\includegraphics[width=10cm]{../../Descargas/UPCDLZMDG5783-logo.png} 
\end{figure}

\newpage

\section{Marco teorico}
\subsection{Tiristores}
El tiristor es un semiconductor de potencia que se utiliza como un interruptor, ya sea para conducir o interrumpir la corriente electrica, a este componente se le conoce como de potencia porque se utilizan para manejar grandes cantidades de corriente y de voltaje, a comparacion de los otros semicondutores que manejan cantidades relativamente bajas.

Cuando se habla de tiristores comunmente se les cataloga como un SRC (Sillicon Controlled Rectifier), pero esto no es del todo correcto ya que este tipo es el mas popular y conocido pero no es el unico que existe.

\begin{figure}[h!]
\centering
\includegraphics[width=12cm]{../../Descargas/tiristor.jpg} 
\end{figure}

\subsection{Funcionamiento de un tiristor}
Los tiristores estan formados por 3 terminales, un anodo, un catodo y una compuerta o mejor conocida como "gate", el funcionamiento se asemeja al de un relevador o un interruptor mecanico. Cuando se le aplica corriente a la terminal gate este se activa y obtiene la caracteristica de dejar pasar la electricidad.

\begin{figure}[h!]
\centering
\includegraphics[width=15cm]{../../Descargas/Funcion-tiristor.jpg} 
\end{figure}
\newpage

\section{Activacion con tiristor en convertidor CA-CD y CA-CA}
\subsection{Funcionamiento}
Este tipo de convertidor nos proporciona una frecuencia de salida rectificada (casi constante) de valor Vm, donde Vm es igual al valor pico del voltaje de entrada.

Recordemos que los tiristores son el equivalente de los interruptores mecanicos, ya que son capaces de dejar pasar o bloquear por completo el paso de la corriente parecidos a un relay.

Tambien los tiristores se llegan a utilizar para obtener diferentes resultados en cuanto a la frecuencia que se quiera obtener como la media onda monofasica, onda completa monofassica, media onda trifasica, onda completa trifasica.

\begin{figure}[h!]
\centering
\includegraphics[scale=1]{../../Descargas/circuitoCA-CD.png} 
\end{figure}

\section*{Conclusion}
Como conclusion en esta tarea, puedo decir que en cuanto a investigacion se refiere, los tiristores no difieren demasiado de los relevadores o de los relays, ya que su funcion es practicamente parecida o al menos es a lo que se llegue a entender, seria interesante llevar esto a la practica y lograr ver su funcionamiento, pero con que nos expliquen para lograr asi tener aun mas entendimiento en cuanto al tema.


\end{document}